\documentclass[12pt]{eccomas-2022-_abstract}

%\usepackage{graphicx}
%\usepackage{amsmath}
%\usepackage{amsfonts}
%\usepackage{amssymb}

\title{ABSTRACT TITLE}

\author{First A. Author$^{1}$, Second B. Author$^{2}$ and Third C. Author$^{3}$}

\address{$^{1}$ Affiliation, Postal Address, E-mail address and URL
\and
$^{2}$ Affiliation, Postal Address, E-mail address and URL
\and
$^{3}$ Affiliation, Postal Address, E-mail address and URL}

\begin{document}

\noindent {\bf Keywords}: {\it Instructions, Multiphysics Problems, Applications, Computing Methods}
\vskip0.5cm

Authors are invited to submit electronically a one page abstract through the ECCOMAS 2022 Congress web site before December 10th, 2021. Abstracts should outline the main features, results and conclusions as well as their general significance, and contain relevant references.

The Abstract can be submitted directly in its final form. Authors will have the possibility of replacing the file by an updated version after the acceptance notification.

The Abstract should be written following the format. The file must be converted to Portable Document Format (PDF) before submission through the Congress web site. Other formats are not accepted by the system.

The Abstract must be written in English following this format template. It must contain the full name and full address of author/s. In the case of joint authorships, the name of the author who will actually present the paper at the Conference should be indicated with an asterisk. 

\emph{Contributions can only be accepted on the understanding that they will be presented at the Congress.}

For any question, please contact the ECCOMAS 2022 Conference Secretariat.\\
E-mail: ECCOMAS2022@cimne.upc.edu 


\begin{thebibliography}{99}
\bibitem{Zienkiewicz}  O.C. Zienkiewicz, O.C. and  R.L. Taylor,  \textit{The finite element method}. 6th Edition, Elsevier, 2005.
\bibitem{OnCe93} E. O\~{n}ate and M. Cervera, Derivation of thin plate bending elements with one degree of freedom per node.
\textit{Engng. Comput.}, Vol. \textbf{10}, pp. 543--561, 1993.
\end{thebibliography}

\end{document}


